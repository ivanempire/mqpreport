\section*{Method}
The method that was used to design the reactor was anything but linear. After getting acquainted with MCNP, we started creating basic core and its shell geometries. This was done in order to get a feel for the sizes that we will be working with throughout the project. Primarily, we wanted to make sure that the core would be at a critical state during operation. This condition is determined by MCNP's $k$ coefficient calculation. If the value of the coefficient is greater than 1, then the core is supercritical. Exactly equal to 1 - the core is simply critical. Anything below 1 causes the reaction to halt since not enough particles are interacting for the fission to move forward. We wanted to use a rectangular core structure (or perhaps cubic) that consisted of Uranium and graphite. A pure-Uranium core would cause serious issues since at that point the system could be turned into a weapon. The concentration was kept just under the minimum threshold. The graphite takes up the majority of the core's concentration to act a neutron dampener. Its placement in between the Uranium blocks allows for particle absorption and overall reaction regulation.

After establishing the general structure of the core to input into MCNP, we turned our attention to the shell and containment unit. We arbitrarily took a radius of $r=100cm$ for the immediate shell surrounding the core. Looking at that shell, we considered 3 materials with which to fill (or not) the space. The first choice is vacuum, followed by water and then graphite. Each of these has their merits:
\begin{itemize}
\item Vacuum - the particles will not have anything to interact with during their emission. This makes the geometry easier to work with as the collisions will take place between the end of the shell and other emitted particles.
\item Water - easier to work with when matters come to thermal regulation. This is outside the scope of the project, however, a possible heat exchanger could be added to the shell to cool off the water. The 2 closed loop system could, in theory, be simply plugged into a building's water supply.
% HERE
\item Graphite - 
\end{itemize}

Various size configurations were tested in order to achieve the optimum criticality, and the results can be seen in table X %HERE

From there onwards, the neutron flux chamber had to be added to the design. At this point, the approach was slightly altered. Instead of having the core placed directly in the center of the spherical chamber, it was moved to the edge that would serve as the neutron emission path. As a result, the neutrons that will be detected coming out of the reactor would be emitted directly from one side of the cubic core. Those neutrons would immediately be used for research purposes. However, the other 5 sides will be interacting 