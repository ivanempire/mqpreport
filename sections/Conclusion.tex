\section{Conclusion}
\label{sec:conclusion}

The goal of this project was to design a reactor concept that would generate a high neutron flux in a small form factor, something that could fit in a room optimistically. The flux of GOFR exceeds other research reactors, by a factor of a 180 compared to the UMass Lowell reactor. The final dimensions of GOFR are about 2 meters on all sides. The size of the Lowell reactor is not publicly available but it is a swimming pool type reactor, which range from about 60 to 365 cubic meters. GOFR is 8 cubic meters, which means that it could fit in the bedroom of a rundown Brooklyn apartment, with a twin bed, and lots of shelf space if no city statutes are violated and the room is somewhat squarish. By those metrics, the goal was reached.\\

Cooling is not yet accounted for, and that will add not insignificant volume, but that should able to be done compactly enough that it still retains its appeal. The flux is somewhat artificially high due to not being collimated but that should not meaningfully decrement the neutron flux. The primary utility and value of this design is the somewhat novel approach to generating a high neutron flux, which was proven to be useful and consequential.\\

The design can be used in a variety of labs due to its small size, lower cost, and competitive neutron flux. This will fill the niche of cheap, affordable neutron sources for smaller research labs with limited funds or space and allow research into various kinds of neutron imaging, relieving the pressure on larger neutron sources and allowing for rapid experimental development. Research will be done by more people and more quickly. Furthermore, this could aid in medical isotope production, and perhaps even provide small amounts of power.\\