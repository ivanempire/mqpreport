\section{Future work}
\label{sec:future}

There are many areas where this project could be extended or improved. Those are, but not limited to, simulation of a collimator, better characterization of the beam, thorium conversion, thorium breeder conversion, better alloys in the core, analysis of the core behavior, thermal simulation and regulation, cost analysis, actual nuclear engineering, and building the reactor.

\subsection{Collimation and Characterization}

The beam produced currently has a slightly higher intensity in the center, if measured at the focal point, and is uniform if measured far enough away from the focal point. However, the vector of travel of the neutrons is not parallel and, if a lower energy beam is desired, there is no way to modulate the neutron energy. These discrepancies in trajectories can be remedied by a collimator. It will thermalize and collimate the beam, making it useful for imaging. This, however, needs to be simulated to ensure that the benefits of this design of reactor are sustained and the resultant beam is useful.

\subsection{Core Advancements}

This reactor core, specifically the use of zirconium hydride, was inspired by the TRIGA reactors. However, in this case, the alloy seemed to have made little difference in the performance of the reactor, which was surprising. Further research into what is happening and ensuring that the benefits are realized is important to maximize the utility of this reactor. Furthermore, the amount and type of alloy to use was a bit difficult to determine and requires more research to be done optimally.\\

Converting the fuel used to thorium would be a good advancement on this reactor as it would be consistent with the ideals of GenIV reactor design. Furthermore, if this thorium reactor could be used to breed more thorium, that would be a efficient use of resources and have applications in broader areas.

\subsection{Engineering Concerns}

Engineering is outside the scope of this project but a few concerns were kept in mind during the design of GOFR. The aim was for GOFR to passively reduce criticality as the temperature reached unsafe levels. Therefore, for sustained high level power, adequate cooling is needed for optimal behavior. To that end, the design hoped to use the aluminum to quickly and efficiently transport heat from the core to the graphite shell. In the shell, there would be a matrix of pipes for coolant to flow through and/or coolant would flow over the surface of the shell, conducting heat away. The cooling system in mind during the project was a compact magnetic refrigeration system, using a gallium alloy in a magnetic field to magnetocalorically cool the reactor. The energy for this could potentially be subsidized by the heat of the reactor itself. Alternatively, since this reactor is meant to be housed in a relatively small space on a academic campus, this could augment HVAC systems as a source of heat and energy. The main idea of the project is to not waste neutrons. This would be a poetic expansion into not wasting energy from neutrons.\\

Furthermore, this was not designed by nuclear engineers. There are many concerns, caveats, regulations, and other such things that we are wholly ignorant to. This requires through and extensive development by people that are more qualified with software more powerful than a Monte Carlo simulation done on overly simplistic models. There are wear, thermal expansion, refueling, safety, cooling, and cost considerations to evaluate and address. This is only the beginning of a long process.\\

Hopefully, one day, this will be realized in an actual reactor and be used by scientists to make the world a better place.