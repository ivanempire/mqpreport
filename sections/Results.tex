\label{sec:results}
\section{Results}

The goal of the project is to test the feasibility of a room-sized reactor for research purposes. The room-size is key here for initial criticality calculations. The shell radius was first taken to be $r=10cm$, resulting in a reactor just over 20cm on its length, width and height. However, the core was not critical enough at these dimensions, and hence the radius was increased to $r=100cm$. With this, a pure Uranium-235 core yielded values that are presented in table \ref{tab:pure}.
\begin{table}[!htbp]
\centering
\caption{Theoretical criticality coefficients.}
\label{tab:pure}
\begin{tabular}{|c|c|c|c|c|}
\hline
Length (cm) & Graveyard radius (cm) & Composition & Submersion & Coefficient \\
\hline
10 & 10 & Pure U-235  & Vacuum & 0.74311 \\
\hline
20 & 10 & Pure U-235  & Vacuum & 1.30614 \\
\hline
10 & 100 & Pure U-235  & Water & 0.99284 \\
\hline
12 & 100 & Pure U-235  & Water & 1.12452 \\
\hline
\end{tabular}
\end{table}
These results are extremely theoretical, however, and were used to get a general estimate for various criticalities given the dimensions of the core, as well as the material in which it is submerged.

In order to obtain practical results that can be used in real-world applications of the reactor, the percentage composition of the core had to be changed. Most of the core is now Graphite, at a composition level of 81\%. The rest is Uranium-235, with a percentage composition of 19\%. Table \ref{tab:nonpure} presents criticalities with the proper composition.

\begin{table}[!htbp]
\centering
\caption{Practical criticality coefficients.}
\label{tab:nonpure}
\begin{tabular}{|c|c|c|c|c|}
\hline
Length (cm) & Graveyard radius (cm) & Composition & Submersion & Coefficient \\
\hline
10 & 100 & 19\% - 81\% split  & Water & 1.0544 \\
\hline
12 & 100 & 19\% - 81\% split  & Water & 1.2042 \\
\hline
24 & 100 & 19\% - 81\% split  & Water & 1.7077 \\
\hline
25 & 100 & 19\% - 81\% split  & Water & 1.7125 \\
\hline
\end{tabular}
\end{table}
