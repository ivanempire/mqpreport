\section*{Design}
\subsection*{Core}
\indent We decided to take a bottom-up approach with the overall reactor design implementation. Looking at the core in figure X
%HERE
one can see that the overall geometry consists of 6 uranium blocks, separated by graphite. This was simplified to a rectangular block of mixed compositions. The majority of the core, 81\% to be exact, is made up of Graphite. The other 19\% are Uranium-235. This is just below the weapons-grade threshold of 20\%. We believe that through this mixed composition the generated particles will interact in a sustainable way inside the chamber, keeping the neutron flux at a controlled rate. If the neutrons become too energized, then they will not interact with the core and the reaction will slow down. The whole process is self-regulating in order to prevent chain reactions from happening.\\
\newline
\indent The dimensions of the rectangular core were determined through a series of criticality tests. The main determinant of this was the $k$ coefficient. It is a measure in MCNP that allows one to see if a core will be critical or not. If the value is at exactly $k=1$, the core is critical. Anything greater than this will result in a supercritical core, and anything less will mean that the core is sub-critical. The most ideal value obtained was $k=1.0544$, which was seen with a core of length $l=10cm$. The cubic core was submerged in the center of a water-filled container, the radius of which was $r=100cm$. From these dimensions, one can calculate the volumetric ratio between the core volume and the chamber:
\begin{align*}
R_{V} &= \frac{V_c}{V_o}\\
&= \frac{1000}{4.19*10^6}\\
&= 0.023\%
\end{align*}
\subsection*{Output chamber}